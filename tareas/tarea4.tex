\documentclass[12pt]{article}

\addtolength{\hoffset}{-2.25cm}
\addtolength{\textwidth}{4.5cm}
\addtolength{\voffset}{-2.5cm}
\addtolength{\textheight}{5cm}
\setlength{\parskip}{0pt}
\setlength{\parindent}{15pt}

\usepackage{amsthm}
\usepackage{amsmath}
\usepackage{amssymb}
\usepackage{enumitem}
\usepackage[colorlinks = true, linkcolor = blue, citecolor = blue, final]{hyperref}

\usepackage{graphicx}
\usepackage{multicol}
\usepackage{ marvosym }
\usepackage{wasysym}
\usepackage{tikz}
\usepackage{xcolor} 
\usepackage{CJKutf8}
\usepackage{tabularx}
\usepackage{float}
\usepackage{pgfplots}
\usepackage{fancyhdr}
\usepackage{amsfonts}
\usepackage{physics}
\usepackage{amsmath}
\usetikzlibrary{patterns}

\newcommand{\subscript}[2]{$#1 _ #2$}
\newcommand{\ds}{\displaystyle}
\newcommand\N{\ensuremath{\mathbb{N}}}
\newcommand\K{\ensuremath{\mathbb{K}}}
\newcommand\R{\ensuremath{\mathbb{R}}}
\newcommand\Z{\ensuremath{\mathbb{Z}}}
\renewcommand\O{\ensuremath{\emptyset}}
\newcommand\Q{\ensuremath{\mathbb{Q}}}
\newcommand\C{\ensuremath{\mathbb{C}}}
\DeclareMathOperator{\spn}{span}

\pgfplotsset{compat=1.18} 
\setlength{\parindent}{0in}

\pagestyle{empty}

\begin{document}

\thispagestyle{empty}

{\scshape Algebra Lineal} \hfill {\scshape \large Tarea IV} \hfill {\scshape Victor Ortega}
 
\smallskip

\hrule

\bigskip

\bigskip

\theoremstyle{definition}
\newtheorem*{definition}{Definición}

\theoremstyle{definition}
\newtheorem*{remark}{Observación}

\theoremstyle{definition}
\newtheorem*{dem}{Demostración}

\theoremstyle{definition}
\newtheorem*{notation}{Notación}

\theoremstyle{definition}
\newtheorem*{theorem}{Teorema}

\theoremstyle{definition}
\newtheorem*{lema}{Lema}

\theoremstyle{remark}
\newtheorem*{observation}{Observación}

\theoremstyle{remark}
\newtheorem*{example}{Ejemplo}


\begin{enumerate} 

\item Sean $V$ y $W$ dos espacios vectoriales sobre un campo $\K$, se dice entonces que $T : V \to W$ es una transformación lineal, $\Leftrightarrow$

$$T \left( \sum_{i=1}^{n} \lambda_i {\vec{v}}_{i} \right) =  \sum_{i=1}^{n}\lambda_i  T\left( {\vec{v}}_{i} \right)$$

con $\lambda_1, ..., \lambda_n \in \K$ y ${\vec{v}}_{1}, ..., {\vec{v}}_{n} \in V$

\begin{proof}
$\Rightarrow$

Supongamos que $T$ es lineal, por lo que abre sumas y saca escalares. 

Si $n=1$, por las dos propiedades de la definición de transformación lineal:

$$T(\lambda_1 {\vec{v}}_{i} ) = \lambda_1 \cdot T({\vec{v}}_{i})$$

Supongamos el resultado cierto para $n=k$

Ahora, veamos que es cierto para $n = k + 1$

$$T \left( \sum_{i=1}^{n} \lambda_i {\vec{v}}_{i} \right) = T \left( \lambda_1 {\vec{v}}_{1} \right) + ... + T\left( \lambda_k {\vec{v}}_{k} \right) + T\left( \lambda_{k+1} {\vec{v}}_{k+1} \right) $$

Como estamos suponiendo el resultado cierto para $k$, se tiene que,

$$ \sum_{i=1}^{k}\lambda_i  T\left( {\vec{v}}_{i} \right)  + T\left( \lambda_{k+1} {\vec{v}}_{k+1} \right) = \sum_{i=1}^{k}\lambda_i  T\left( {\vec{v}}_{i} \right) + \lambda_{k+1}  \cdot T\left( {\vec{v}}_{k+1}  \right) = \sum_{i=1}^{n}\lambda_i  T\left( {\vec{v}}_{i} \right)$$

$\Leftarrow$

Suponga que $T \left( \sum\limits_{i=1}^{n} \lambda_i {\vec{v}}_{i} \right) =  \sum\limits_{i=1}^{n}\lambda_i  T\left( {\vec{v}}_{i} \right)$. Veamos que $T$ es transformación lineal. Es evidente ver que abre sumas y saca escalares. Pero esta es la definición de transformación lineal. 
\end{proof}

\end{enumerate}

\end{document}