\documentclass[12pt]{article}

\addtolength{\hoffset}{-2.25cm}
\addtolength{\textwidth}{4.5cm}
\addtolength{\voffset}{-2.5cm}
\addtolength{\textheight}{5cm}
\setlength{\parskip}{0pt}
\setlength{\parindent}{15pt}

\usepackage{amsthm}
\usepackage{amsmath}
\usepackage{amssymb}
\usepackage{enumitem}
\usepackage[colorlinks = true, linkcolor = blue, citecolor = blue, final]{hyperref}

\usepackage{graphicx}
\usepackage{multicol}
\usepackage{ marvosym }
\usepackage{wasysym}
\usepackage{tikz}
\usepackage{xcolor} 
\usepackage{CJKutf8}
\usepackage{tabularx}
\usepackage{float}
\usepackage{pgfplots}
\usepackage{fancyhdr}
\usepackage{amsfonts}
\usepackage{physics}
\usepackage{amsmath}
\usetikzlibrary{patterns}

\newcommand{\subscript}[2]{$#1 _ #2$}
\newcommand{\ds}{\displaystyle}
\newcommand\N{\ensuremath{\mathbb{N}}}
\newcommand\K{\ensuremath{\mathbb{K}}}
\newcommand\R{\ensuremath{\mathbb{R}}}
\newcommand\Z{\ensuremath{\mathbb{Z}}}
\renewcommand\O{\ensuremath{\emptyset}}
\newcommand\Q{\ensuremath{\mathbb{Q}}}
\newcommand\C{\ensuremath{\mathbb{C}}}
\DeclareMathOperator{\spn}{span}

\pgfplotsset{compat=1.18} 
\setlength{\parindent}{0in}

\pagestyle{empty}

\begin{document}

\thispagestyle{empty}

{\scshape Algebra Lineal} \hfill {\scshape \large Tarea I} \hfill {\scshape Victor Ortega}
 
\smallskip

\hrule

\bigskip

\bigskip

\theoremstyle{definition}
\newtheorem*{definition}{Definición}

\theoremstyle{definition}
\newtheorem*{dem}{Demostración}

\theoremstyle{definition}
\newtheorem*{notation}{Notación}

\theoremstyle{definition}
\newtheorem*{theorem}{Teorema}

\theoremstyle{definition}
\newtheorem*{lema}{Lema}

\theoremstyle{remark}
\newtheorem*{observation}{Observación}

\theoremstyle{remark}
\newtheorem*{example}{Ejemplo}


\begin{enumerate} 

\item Sea $V$ un $\K$-espacio vectorial y $W_1 \leq V$ y $W_2 \leq V$ subespacios de $V$. Demuestre que $W_1 \cup W_2 \leq V \Leftrightarrow W_1 \subseteq W_2$ o $W_2 \subseteq W_1$

\begin{proof}

    $\Leftarrow$

    Supongamos que $W_1 \subseteq W_2$ o $W_2 \subseteq W_1$ P.D. $W_1 \cup W_2 \leq V$
    \begin{align*}
          \Rightarrow  W_1 = W_1 \cup W_2   & & \text{o} & &  W_2 = W_1 \cup W_2 
    \end{align*}

    Como $W_1 \leq V$ y $W_2 \leq V \Rightarrow W_1 \cup W_2 \leq V$

    $\Rightarrow$

    Supongamos que $W_1 \cup W_2 \leq V$ P.D.  $W_1 \subseteq W_2$ o $W_2 \subseteq W_1$

    Supongamos que $W_2 \nsubseteq W_1$ P.D. $W_1 \subseteq W_2$ 
    
    Sea $\vec{v} \in W_1$. Como  $W_2 \nsubseteq W_1 \Rightarrow \exists \: \vec{w} \in W_2 \backepsilon \vec{w} \notin W_1 $. Por definición de $\cup$ s.t.q.
    \begin{align*}
          \Rightarrow  \vec{v} \in W_1 \subseteq  W_1 \cup W_2  & & \text{y} & &  \vec{w} \in W_2 \subseteq  W_1 \cup W_2 
    \end{align*}

    Como $W_1 \cup W_2  \leq V$ subespacio s.t.q.


    \begin{equation*}
        \vec{v} + \vec{w} \in W_1 \cup W_2
    \end{equation*}
    \begin{align*}
          \Rightarrow  \vec{v} + \vec{w} \in W_1  & & \text{o} & &  \vec{v} + \vec{w} \in W_2
    \end{align*}

    Note que, como $W_1$ es un subespacio s.t.q.

    \begin{equation*}
        (\vec{v} + \vec{w}) + (-\vec{v} ) \in W_1
    \end{equation*}

    Pero esto, por hipótesis, es imposible, por lo tanto s.t.q. $\vec{v} + \vec{w} \in W_2$

    \begin{equation*}
        \Rightarrow (\vec{v} + \vec{w}) + (-\vec{w} )  \in W_2
    \end{equation*}
    \begin{equation*}
        \Rightarrow \vec{v} \in W_2
    \end{equation*}

    $\therefore $ como $\vec{v}$ arbitrario $W_1 \subseteq W_2$

    Sin perdida de generalidad, la prueba para $W_2 \subseteq W_1$ es análoga. Cuando $W_2 = W_1$  la demostración es trivial.
\end{proof}

\item Demostrar que $\vec{v}+W \leq V \Rightarrow \vec{v} \in W$

\begin{proof}
    Supongamos que $\vec{v}+W \leq V$

    Como $\vec{v}+W$ es un subespacio contiene a $\vec{0}$ $\Rightarrow \exists  \: \vec{w} \in \vec{v}+W   \backepsilon \vec{v} + \vec{w} = 0$

    $\Rightarrow - \vec{v} \in W$ Como $\vec{v}+W$ es un subespacio $\Rightarrow \forall \: \vec{w} \in \vec{v}+W$ y $\forall \: \lambda \in \K \Rightarrow \lambda \cdot \vec{w} \in \vec{v}+W$

    $\Rightarrow$ con $\lambda = -1 \Rightarrow (-1) \cdot -\vec{v} \in \vec{v} + W$

    $\therefore \vec{v} \in W$
\end{proof}

\item Sea ($V,+,\cdot$) un $\K$-espacio vectorial y $S \subseteq V$ un conjunto no vacío de vectores de $V$. Demuestre que

\begin{equation*}
    \langle S \rangle = \cap \: \{ W \mid W \leq V \backepsilon S \subseteq W \}
\end{equation*}

\begin{proof}

    $\subseteq$

    Demostramos en clase que  $\langle S \rangle$ es el subespacio más chico quecontiene al conjunto, por lo que es uno de los conjuntos $W$

    $\supseteq$
    
    Sea $\vec{v} \in \cap \: \{ W \mid W \leq V \backepsilon S \subseteq W \}$

    Por lo que $\vec{v} $ pertenece a cada subespacio de $V$ que contiene a $S$. Pero $\langle S \rangle$ es uno de estos subespacios. $\Rightarrow \vec{v} \in  \langle S \rangle $

    $$\Rightarrow \cap \: \{ W \mid W \leq V \backepsilon S \subseteq W \} \subseteq \langle S \rangle$$

    $\therefore \langle S \rangle = \cap \: \{ W \mid W \leq V \backepsilon S \subseteq W \}$ 

\end{proof}


\end{enumerate}

\end{document}
