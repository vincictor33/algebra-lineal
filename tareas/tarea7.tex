\documentclass[12pt]{article}

\addtolength{\hoffset}{-2.25cm}
\addtolength{\textwidth}{4.5cm}
\addtolength{\voffset}{-2.5cm}
\addtolength{\textheight}{5cm}
\setlength{\parskip}{0pt}
\setlength{\parindent}{15pt}

\usepackage{amsthm}
\usepackage{amsmath}
\usepackage{amssymb}
\usepackage{enumitem}
\usepackage[colorlinks = true, linkcolor = blue, citecolor = blue, final]{hyperref}

\usepackage{graphicx}
\usepackage{multicol}
\usepackage{ marvosym }
\usepackage{wasysym}
\usepackage{tikz}
\usepackage{xcolor} 
\usepackage{CJKutf8}
\usepackage{tabularx}
\usepackage{float}
\usepackage{pgfplots}
\usepackage{fancyhdr}
\usepackage{amsfonts}
\usepackage{physics}
\usepackage{amsmath}
\usetikzlibrary{patterns}

\newcommand{\subscript}[2]{$#1 _ #2$}
\newcommand{\ds}{\displaystyle}
\newcommand\N{\ensuremath{\mathbb{N}}}
\newcommand\K{\ensuremath{\mathbb{K}}}
\newcommand\R{\ensuremath{\mathbb{R}}}
\newcommand\Z{\ensuremath{\mathbb{Z}}}
\renewcommand\O{\ensuremath{\emptyset}}
\newcommand\Q{\ensuremath{\mathbb{Q}}}
\newcommand\C{\ensuremath{\mathbb{C}}}
\DeclareMathOperator{\spn}{span}

\pgfplotsset{compat=1.18} 
\setlength{\parindent}{0in}

\pagestyle{empty}

\begin{document}

\thispagestyle{empty}

{\scshape Algebra Lineal} \hfill {\scshape \large Tarea VII} \hfill {\scshape Victor Ortega}
 
\smallskip

\hrule

\bigskip

\bigskip

\theoremstyle{definition}
\newtheorem*{definition}{Definición}

\theoremstyle{definition}
\newtheorem*{remark}{Observación}

\theoremstyle{definition}
\newtheorem*{dem}{Demostración}

\theoremstyle{definition}
\newtheorem*{notation}{Notación}

\theoremstyle{definition}
\newtheorem*{theorem}{Teorema}

\theoremstyle{definition}
\newtheorem*{lema}{Lema}

\theoremstyle{definition}
\newtheorem*{corollary}{Corollary}

\theoremstyle{remark}
\newtheorem*{observation}{Observación}

\theoremstyle{remark}
\newtheorem*{example}{Ejemplo}


\begin{theorem}
    Sean $T: U \to V$, $S : V \to W$, y $R : W \to X$ tres funciones cualesquiera $\Rightarrow$

    $$R(ST)= (RS)T$$
\end{theorem}

\begin{proof}
    Por definición de composición, ambas funciones $R(ST)$ y $(RS)T$ tienen dominio $U \Rightarrow \: \forall \: x \in U$

    $$[R(ST)](x)= R[(ST)(x)]= R[S[T(x)]]$$
    $$[(RS)(T)](x)=(RS)[T(x)]=R[S[T(x)]]$$

    $\therefore R(ST)= (RS)T$
\end{proof}

\begin{corollary}
    Como esto es cierto para cualquier función, es cierto, en particular, para transformaciones lineales, lo que demuestra lo pedido en la tarea. 
\end{corollary}

\begin{theorem}
    Sea $A \in M_{m \times n}(F)$. Obtenemos la transformación llamada multiplicación por la izquierda $L_A :\K^n \to \K^m$ definida por  $L_A(x)=Ax \Rightarrow L_A$ es lineal, y si $\beta$ y $\gamma$ son bases ordenadas de $\K^n$ y $\K^m$ se cumple que

    $$L_{\lambda A + B} = \lambda L_A + L_B$$
\end{theorem}

\begin{proof}
    Ya demostramos que 

    \begin{equation} \label{eq1}
        {[L_A]}_{\beta}^{\gamma} = A
    \end{equation}

    y que 

    \begin{equation} \label{eq2}
        L_A = L_B \iff A = B
    \end{equation}

    Entonces, por \ref{eq1}, sabemos que

    $$ {[L_{\lambda A + B}]}_{\beta}^{\gamma} = \lambda A + B$$

    Además

    $${[\lambda A + B]}_{\beta}^{\gamma} =  \lambda {[L_A]}_{\beta}^{\gamma} +  {[L_B]}_{\beta}^{\gamma} = \lambda A + B$$

    De \ref{eq2}, se sigue que 
    
    $$L_{\lambda A + B} = \lambda L_A + L_B$$
\end{proof}
\end{document}