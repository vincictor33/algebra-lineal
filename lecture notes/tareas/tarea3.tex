\documentclass[12pt]{article}

\addtolength{\hoffset}{-2.25cm}
\addtolength{\textwidth}{4.5cm}
\addtolength{\voffset}{-2.5cm}
\addtolength{\textheight}{5cm}
\setlength{\parskip}{0pt}
\setlength{\parindent}{15pt}

\usepackage{amsthm}
\usepackage{amsmath}
\usepackage{amssymb}
\usepackage{enumitem}
\usepackage[colorlinks = true, linkcolor = blue, citecolor = blue, final]{hyperref}

\usepackage{graphicx}
\usepackage{multicol}
\usepackage{ marvosym }
\usepackage{wasysym}
\usepackage{tikz}
\usepackage{xcolor} 
\usepackage{CJKutf8}
\usepackage{tabularx}
\usepackage{float}
\usepackage{pgfplots}
\usepackage{fancyhdr}
\usepackage{amsfonts}
\usepackage{physics}
\usepackage{amsmath}
\usetikzlibrary{patterns}

\newcommand{\subscript}[2]{$#1 _ #2$}
\newcommand{\ds}{\displaystyle}
\newcommand\N{\ensuremath{\mathbb{N}}}
\newcommand\K{\ensuremath{\mathbb{K}}}
\newcommand\R{\ensuremath{\mathbb{R}}}
\newcommand\Z{\ensuremath{\mathbb{Z}}}
\renewcommand\O{\ensuremath{\emptyset}}
\newcommand\Q{\ensuremath{\mathbb{Q}}}
\newcommand\C{\ensuremath{\mathbb{C}}}
\DeclareMathOperator{\spn}{span}

\pgfplotsset{compat=1.18} 
\setlength{\parindent}{0in}

\pagestyle{empty}

\begin{document}

\thispagestyle{empty}

{\scshape Algebra Lineal} \hfill {\scshape \large Tarea III} \hfill {\scshape Victor Ortega}
 
\smallskip

\hrule

\bigskip

\bigskip

\theoremstyle{definition}
\newtheorem*{definition}{Definición}

\theoremstyle{definition}
\newtheorem*{remark}{Observación}

\theoremstyle{definition}
\newtheorem*{dem}{Demostración}

\theoremstyle{definition}
\newtheorem*{notation}{Notación}

\theoremstyle{definition}
\newtheorem*{theorem}{Teorema}

\theoremstyle{definition}
\newtheorem*{lema}{Lema}

\theoremstyle{remark}
\newtheorem*{observation}{Observación}

\theoremstyle{remark}
\newtheorem*{example}{Ejemplo}


\begin{enumerate} 

\item $V / W$ es un espacio vectorial, llamado $V$ módulo $W$. Definido con las siguientes operaciones.

\begin{align*}
    \oplus : ( \vec{v} + W ) + ( \vec{u} + W ) = ( \vec{v} + \vec{u} ) + W  && \text{ y } && \odot : \lambda(\vec{v}+W)= (\lambda \cdot \vec{v}) +W
\end{align*}

\begin{remark}
    En la clase se demostró que eran operaciones bien definidas, el primer inciso de la tarea. Es por ello que no se va a realizar en este entregable, ya que ya se hizo en clase. Sola falta especificar que \(\vec{v}_1-\vec{v}_1^\prime \in W\), se sigue que \(\lambda(\vec{v}_1-\vec{v}_1^\prime)\in W\), o bien, que \(\lambda\vec{v}_1-\lambda\vec{v}_1^\prime \in W\). Por lo tanto \(\lambda(\vec{v}_1+W)=\lambda(\vec{v}_1^\prime+W)\). 
\end{remark}

\begin{proof}
    Notemos que de $V$ son sumamente triviales las leyes asociativas, conmutativas y distributivas. Debemos únicamente demostrar la existencia del neutra aditivo y el inverso aditivo.

    Sea $\vec{v} + W \in V / W$. Veamos que $W = 0 +W$ es el neutro aditivo

    \begin{equation*}
        (\vec{v} + W) + W = (\vec{v} + 0) + W = \vec{v} + W \Rightarrow W + (\vec{v} + W) = (0 + \vec{v}) + W = \vec{v} + W
    \end{equation*}

    Sea $\vec{v} + W \in V / W$. Veamos que $-\vec{v} + W$ es el inverso aditivo. 
        
    \begin{equation*}
        (\vec{v} + W) + (-\vec{v} + W) = (\vec{v} - \vec{v}) + W = 0 + W = W
    \end{equation*}

    Lo que nos da el 0 del espacio, que es $W$, como ya vimos en la primera parte. 

    $\therefore V / W$ es un espacio vectorial sobre el campo $\K$. 
\end{proof}

\begin{remark}
    Para evidenciar que son triviales las demás propiedades, note lo fácil que es demostrar la asociatividad de la suma

    \begin{equation*}
        (\vec{u} + \vec{v}) + W + \vec{w} + W = (\vec{u} + \vec{v}) + \vec{w} + W = \vec{u} (\vec{v} + \vec{w}) + W
    \end{equation*}

    Demostremos también la distribución de multiplicación escalar
    \begin{equation*}
        (\lambda  + \gamma ) \cdot \vec{v} + W = (\lambda  + \gamma )\cdot \vec{v} + W = (\lambda \cdot \vec{v} + \gamma \cdot \vec{v}) + W = (\lambda \cdot \vec{v} + W) + (\gamma \cdot \vec{v} + W) = \lambda \cdot (\vec{v} + W) + \gamma \cdot (\vec{v} + W)
    \end{equation*}

    Es por ello que decimos que las propiedades son triviales.  
\end{remark}
\end{enumerate}

\end{document}