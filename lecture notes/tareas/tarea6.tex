\documentclass[12pt]{article}

\addtolength{\hoffset}{-2.25cm}
\addtolength{\textwidth}{4.5cm}
\addtolength{\voffset}{-2.5cm}
\addtolength{\textheight}{5cm}
\setlength{\parskip}{0pt}
\setlength{\parindent}{15pt}

\usepackage{amsthm}
\usepackage{amsmath}
\usepackage{amssymb}
\usepackage{enumitem}
\usepackage[colorlinks = true, linkcolor = blue, citecolor = blue, final]{hyperref}

\usepackage{graphicx}
\usepackage{multicol}
\usepackage{ marvosym }
\usepackage{wasysym}
\usepackage{tikz}
\usepackage{xcolor} 
\usepackage{CJKutf8}
\usepackage{tabularx}
\usepackage{float}
\usepackage{pgfplots}
\usepackage{fancyhdr}
\usepackage{amsfonts}
\usepackage{physics}
\usepackage{amsmath}
\usetikzlibrary{patterns}

\newcommand{\subscript}[2]{$#1 _ #2$}
\newcommand{\ds}{\displaystyle}
\newcommand\N{\ensuremath{\mathbb{N}}}
\newcommand\K{\ensuremath{\mathbb{K}}}
\newcommand\R{\ensuremath{\mathbb{R}}}
\newcommand\Z{\ensuremath{\mathbb{Z}}}
\renewcommand\O{\ensuremath{\emptyset}}
\newcommand\Q{\ensuremath{\mathbb{Q}}}
\newcommand\C{\ensuremath{\mathbb{C}}}
\DeclareMathOperator{\spn}{span}

\pgfplotsset{compat=1.18} 
\setlength{\parindent}{0in}

\pagestyle{empty}

\begin{document}

\thispagestyle{empty}

{\scshape Algebra Lineal} \hfill {\scshape \large Tarea VI} \hfill {\scshape Victor Ortega}
 
\smallskip

\hrule

\bigskip

\bigskip

\theoremstyle{definition}
\newtheorem*{definition}{Definición}

\theoremstyle{definition}
\newtheorem*{remark}{Observación}

\theoremstyle{definition}
\newtheorem*{dem}{Demostración}

\theoremstyle{definition}
\newtheorem*{notation}{Notación}

\theoremstyle{definition}
\newtheorem*{theorem}{Teorema}

\theoremstyle{definition}
\newtheorem*{lema}{Lema}

\theoremstyle{definition}
\newtheorem*{corollary}{Corollary}

\theoremstyle{remark}
\newtheorem*{observation}{Observación}

\theoremstyle{remark}
\newtheorem*{example}{Ejemplo}


\begin{theorem}
    Sea $V = P[\R]$ y para $j \geqslant 1$ definimos $T_j(f(x)) = f^{(j)}(x)$, donde $f^{(j)}(x)$ es la $j$-ésima derivada de $f(x) \Rightarrow \: \forall \: n \in \Z^+$ el conjunto $\{ T_1, ..., T_n \}$ es linealmente independiente en $\mathcal{L}(V)$
\end{theorem}

\begin{proof}
    Sea $T_i \in \mathcal{L}(V)$ con $i = 1, ..., n$, un $n \in \N$ fijo pero arbitrario, y $\lambda \in \R$

    $$\Rightarrow \vec{0}_{\mathcal{L}(V)} = \lambda_1 \cdot T_1 + ... + \lambda_n \cdot T_n$$

    Expresamos al cero como combinación lineal de elementos de $\mathcal{L}(V)$. Notemos que como $n \in \Z^+$

    $$T_i(x_n) = \frac{n !}{(n-i)!} x^{n-i} \Rightarrow \vec{0}_{\mathcal{L}(V)} = \lambda_1 \cdot T_1 + ... + \lambda_n \cdot T_n \iff \lambda_1 = ... = \lambda_n = 0$$
    
    $\therefore \{ T_1, ..., T_n \}$ es linealmente independiente en $\mathcal{L}(V)$
\end{proof}

\begin{theorem}
    Obtengamos transformaciones lineales $T, R : \K^2 \to \K^2$ tales que $R \circ T = T_0$, pero  $T \circ R \neq T_0$. Obtenga también matrices $A$ y $B$ tales que $AB = 0$, pero $BA \neq 0$ 
\end{theorem}

\begin{proof}

Definimos la transformación \(T_0\). Si escogemos \(T_0\) como una transformación lineal nula, donde \(T_0(x) = 0\) para todo \(x \in \K^2\), entonces \(T_0\) es representada por la matriz cero:

    \[
    T_0 = \begin{bmatrix}
    0  \\
    0 
    \end{bmatrix}.
    \]
    
     Definamos a $R=L_A$ y $T=L_B$. Donde
     
    \[
    A = \begin{bmatrix}
    0 & 1 \\
    0 & 1
    \end{bmatrix} \quad \quad \quad \quad \quad \quad B = \begin{bmatrix}
    1 & 1 \\
    0 & 0
    \end{bmatrix}.
    \]
    Veamos que  \(R \circ T = T_0 \) 
    \[
    R \circ T = R\left(\begin{bmatrix}
    1 & 1 \\
    0 & 0 
    \end{bmatrix} \begin{bmatrix}
    a \\
    b
    \end{bmatrix} \right) = \begin{bmatrix}
    0 & 1 \\
    0 & 1
    \end{bmatrix} \begin{bmatrix}
    a+b \\
    0
    \end{bmatrix} = \begin{bmatrix}
    0 \\
    0
    \end{bmatrix}.
    \]
    
     Veamos que \(T \circ R \neq T_0\)
    \[
    T \circ R = R\left(\begin{bmatrix}
    0 & 1 \\
    0 & 1
    \end{bmatrix} \begin{bmatrix}
    a \\
    b
    \end{bmatrix} \right) = \begin{bmatrix}
    1 & 1 \\
    0 & 0
    \end{bmatrix} \begin{bmatrix}
    b \\
    b
    \end{bmatrix} = \begin{bmatrix}
    2b \\
    0
    \end{bmatrix}.
    \]
    
     Calculamos \(AB\):
    \[
    AB = \begin{bmatrix}
    0 & 1 \\
    0 & 1
    \end{bmatrix} \cdot \begin{bmatrix}
    1 & 1 \\
    0 & 0
    \end{bmatrix} = \begin{bmatrix}
    0 \cdot 1 + 1 \cdot 0 &  0 \cdot 1 + 1 \cdot 0  \\
    0 \cdot 1 + 0 \cdot 0 & 0 \cdot 1 + 0 \cdot 0
    \end{bmatrix} = \begin{bmatrix}
    0 & 0 \\
    0 & 0
    \end{bmatrix}.
    \]
    
     Ahora calculamos \(BA\):
    \[
    BA = \begin{bmatrix}
    1 & 1 \\
    0 & 0
    \end{bmatrix} \cdot \begin{bmatrix}
    0 & 1 \\
    0 & 1
    \end{bmatrix} = \begin{bmatrix}
    1 \cdot 0 + 1 \cdot 0 & 1 \cdot 1 + 1 \cdot 1 \\
    0 \cdot 0 + 0 \cdot 0 & 0 \cdot 1+ 0 \cdot 1
    \end{bmatrix} = \begin{bmatrix}
    0 & 2 \\
    0 & 0
    \end{bmatrix}.
    \]

\end{proof}

\bigskip

\begin{theorem}
    Sean $V, W, Z$ tres $\K$-espacios vectoriales, y sean $T : V \to W$ y$ R : W \to Z$ transformaciones lineales $\Rightarrow$

    \begin{enumerate}
        \item Si $R \circ T$ es inyectiva $\Rightarrow$ $T$ es inyectiva.
        \item Si $R \circ T$ es sobreyectiva $\Rightarrow$ $R$ es sobreyectiva.
        \item Si $R$ y $T$ son biyectivas $\Rightarrow$ $R \circ T$ es biyectiva.
    \end{enumerate}

    
\end{theorem}

\begin{proof}
        \begin{enumerate}
            \item Supongamos que \(R \circ T\) es inyectiva, y que \(T(v_1) = T(v_2)\) para algunos \(v_1, v_2 \in V \Rightarrow\). 
        \[
        R(T(v_1)) = R(T(v_2)).
        \]
        Como \(R \circ T\) es inyectiva, se sigue que \(v_1 = v_2\). $\therefore$ \(T\) es inyectiva.

        \item Supongamos que $R \circ T = RT$ es sobreyectiva

        $$\Rightarrow \: \forall \: \vec{z} \in Z \: \exists \: \vec{v} \in V \backepsilon RT(\vec{v}) = \vec{z}$$
         $$\Rightarrow \: \forall \: \vec{z} \in Z \: \exists \: T(\vec{v}) \in W \backepsilon RT(T(\vec{v})) = \vec{z}$$

         $\therefore R$ es sobreyectiva

         \item Supongamos que $R$ y $T$ son biyectivas
         $$\Rightarrow \: \forall \: \vec{z} \in Z \: \exists \: \vec{w} \in W \backepsilon R(\vec{w}) = \vec{z}$$
         $$\Rightarrow \: \exists \: \vec{v} \in V \backepsilon T(\vec{v}) = \vec{w} \Rightarrow  RT(\vec{v}) = \vec{z}$$

         Además

         $$RT(\vec{z}) = \vec{0}_Z \Rightarrow T(\vec{u} = \vec{0}_W \Rightarrow \vec{u} = \vec{0}_V $$

         $\therefore RT$ es biyectiva
        \end{enumerate}
    \end{proof}
\end{document}