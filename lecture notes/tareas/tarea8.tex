\documentclass[12pt]{article}

\addtolength{\hoffset}{-2.25cm}
\addtolength{\textwidth}{4.5cm}
\addtolength{\voffset}{-2.5cm}
\addtolength{\textheight}{5cm}
\setlength{\parskip}{0pt}
\setlength{\parindent}{15pt}

\usepackage{amsthm}
\usepackage{amsmath}
\usepackage{amssymb}
\usepackage{enumitem}
\usepackage[colorlinks = true, linkcolor = blue, citecolor = blue, final]{hyperref}

\usepackage{graphicx}
\usepackage{multicol}
\usepackage{ marvosym }
\usepackage{wasysym}
\usepackage{tikz}
\usepackage{xcolor} 
\usepackage{CJKutf8}
\usepackage{tabularx}
\usepackage{float}
\usepackage{pgfplots}
\usepackage{fancyhdr}
\usepackage{amsfonts}
\usepackage{physics}
\usepackage{amsmath}
\usetikzlibrary{patterns}

\newcommand{\subscript}[2]{$#1 _ #2$}
\newcommand{\ds}{\displaystyle}
\newcommand\N{\ensuremath{\mathbb{N}}}
\newcommand\K{\ensuremath{\mathbb{K}}}
\newcommand\R{\ensuremath{\mathbb{R}}}
\newcommand\Z{\ensuremath{\mathbb{Z}}}
\renewcommand\O{\ensuremath{\emptyset}}
\newcommand\Q{\ensuremath{\mathbb{Q}}}
\newcommand\C{\ensuremath{\mathbb{C}}}
\DeclareMathOperator{\spn}{span}

\pgfplotsset{compat=1.18} 
\setlength{\parindent}{0in}

\pagestyle{empty}

\begin{document}

\thispagestyle{empty}

{\scshape Algebra Lineal} \hfill {\scshape \large Tarea VIII} \hfill {\scshape Victor Ortega}
 
\smallskip

\hrule

\bigskip

\bigskip

\theoremstyle{definition}
\newtheorem*{definition}{Definición}

\theoremstyle{definition}
\newtheorem*{remark}{Observación}

\theoremstyle{definition}
\newtheorem*{dem}{Demostración}

\theoremstyle{definition}
\newtheorem*{notation}{Notación}

\theoremstyle{definition}
\newtheorem*{theorem}{Teorema}

\theoremstyle{definition}
\newtheorem*{lema}{Lema}

\theoremstyle{definition}
\newtheorem*{corollary}{Corollary}

\theoremstyle{remark}
\newtheorem*{observation}{Observación}

\theoremstyle{remark}
\newtheorem*{example}{Ejemplo}



\begin{theorem}
    Sen $a,b \in \R$ con $a < b$. Definimos 
    \begin{equation*}
        C([a,b]) := \{ f: [a,b] \to \R \mid \text{ $f$ es continua }\}
    \end{equation*}

    Es conocido que $ C([a,b])$ es un $\R$-espacio vectorial. Más aún, si $f,g \in C([a,b]) \Rightarrow fg : [a,b] \to \R$ definida por

    \begin{equation*}
        (fg)(x) = f(x)g(x)
    \end{equation*}

    con $x \in X$ es también una función continua. También es bien conocido que $ C([a,b]) \subseteq  R([a,b])$, es decir, toda función continua $f:[a,b] \to \R$ es Riemann integrable

    Bajo estas condiciones, veamos que la función

    \begin{equation*}
        \langle \: , \: \rangle : C([a,b]) \times C([a,b]) \to \R
    \end{equation*}

    definida por
    \begin{equation*}
         \langle f , g \rangle = \int_{a}^{b} f(x)g(x) \, dx
    \end{equation*}

    es un producto interior 
\end{theorem}

\begin{proof}
    \begin{enumerate}
        \item Sea $f,g \in C([a,b]) $ arbitrarios
        \begin{equation*}
            \Rightarrow \langle f , g \rangle = \int_{a}^{b} f(x)g(x) \, dx =  \int_{a}^{b} g(x)f(x) \, dx = \langle g , f \rangle
        \end{equation*}
        \item Sean $f,g,h \in C([a,b]) $ cualesquiera funciones
        \begin{equation*}
            \Rightarrow \langle f + g, h \rangle = \int_{a}^{b} (f(x)+g(x))h(x) \, dx = \int_{a}^{b} f(x)h(x)+g(x)h(x) \, dx 
        \end{equation*}
        \begin{equation*}
            = \int_{a}^{b} f(x)h(x)\, dx + \int_{a}^{b} g(x)h(x)\, dx =  \langle f , h \rangle +  \langle  g, h \rangle
        \end{equation*}
        \item Sean $f,g \in  C([a,b])$ y $\lambda \in \R$ arbitrarios
        \begin{equation*}
            \Rightarrow \langle \lambda f, g  \rangle  = \int_{a}^{b} (\lambda f)(x)h(x)\, dx = \int_{a}^{b} \lambda( f(x)h(x))\, dx =  \lambda \int_{a}^{b}  f(x)h(x)\, dx = \lambda \langle  f, g  \rangle 
        \end{equation*}
        \item Sea $f \in  C([a,b])$ arbitrario. Debido a que $f^2 : [a,b] \to \R$ es continua y $\: \forall \: x \in [a,b] \Rightarrow f^2 (x) \geqslant 0$. Así, como la función es una funcional lineal no negativa 

        \begin{equation*}
            \langle f, f \rangle = \int_{a}^{b} f^2(x)\, dx \geqslant \int_{a}^{b} 0 \, dx = 0
        \end{equation*}

        Ahora, notemos que

        \begin{equation*}
            \langle f, f \rangle=0 \iff \int_a^b \abs{f(x)}^2\,dx=0\stackrel{\text{por continuidad}}\Longleftrightarrow \abs{f(x)}^2=0\,\,\text{en}\,\,[a,b]\iff f(x)=0
        \end{equation*}
    \end{enumerate}
    $\therefore \langle \: , \: \rangle$ es un producto interior en $C([a,b])$
\end{proof}

\begin{theorem}
    Demuestre que un producto interior es lineal en la primera entrada
    $$\left\langle \sum_{i=1}^{n} \lambda_i {\vec{v}}_{i}, \vec{w} \right\rangle = \sum_{i=1}^{n}\lambda_i  \left\langle {\vec{v}}_{i}, \vec{w} \right\rangle$$
\end{theorem}

\begin{proof}
    Note que si $T : V \to W$ es una transformación lineal es lineal $\Rightarrow T$ es lineal $\iff T(\lambda \cdot \vec{u} + \vec{v}) = \lambda \cdot T(\vec{u}) + T(\vec{v})$ 

    \begin{enumerate}
        \item $n=1$ Sean $\vec{v}_1, \vec{w} \in V$ un $\K$-espacio vectorial, y $\lambda_1 \in \K$. Por definición de producto interior, tenemos que 

        $$\left\langle \lambda_1 \vec{v}_1, \vec{w} \right\rangle = \lambda_1 \left\langle \vec{v}_1, \vec{w} \right\rangle.$$

        \item Supongamos el resultado cierto para $n$

        $$\left\langle \sum_{i=1}^{n} \lambda_i \vec{v}_i, \vec{w} \right\rangle = \sum_{i=1}^{n} \lambda_i \left\langle \vec{v}_i, \vec{w} \right\rangle$$

        \item Veamos que se cumple el teorema para $n+1$. Note que 

        $$\sum_{i=1}^{n+1} \lambda_i \vec{v}_i = \left( \sum_{i=1}^{n} \lambda_i \vec{v}_i \right) + \lambda_{n+1} \vec{v}_{n+1}$$

        Aplicamos el producto interior a esta suma:

        $$\left\langle \sum_{i=1}^{n+1} \lambda_i \vec{v}_i, \vec{w} \right\rangle = \left\langle \left( \sum_{i=1}^{n} \lambda_i \vec{v}_i \right) + \lambda_{n+1} \vec{v}_{n+1}, \vec{w} \right\rangle$$

        Por la primer propiedad del producto interior s.t.q.

        $$\left\langle \left( \sum_{i=1}^{n} \lambda_i \vec{v}_i \right) + \lambda_{n+1} \vec{v}_{n+1}, \vec{w} \right\rangle = \left\langle \sum_{i=1}^{n} \lambda_i \vec{v}_i, \vec{w} \right\rangle + \left\langle \lambda_{n+1} \vec{v}_{n+1}, \vec{w} \right\rangle$$

        Aplicando la hipótesis de inducción, y la segunda propiedad de la definición, s.t.q.

        $$\left\langle \sum_{i=1}^{n} \lambda_i \vec{v}_i, \vec{w} \right\rangle + \left\langle \lambda_{n+1} \vec{v}_{n+1}, \vec{w} \right\rangle = \sum_{i=1}^{n} \lambda_i \left\langle \vec{v}_i, \vec{w} \right\rangle + \lambda_{n+1} \left\langle \vec{v}_{n+1}, \vec{w} \right\rangle$$

        Y juntando los escalares, y usando la primro propiedad del producto interior

        $$\sum_{i=1}^{n} \lambda_i \left\langle \vec{v}_i, \vec{w} \right\rangle + \lambda_{n+1} \left\langle \vec{v}_{n+1}, \vec{w} \right\rangle = \sum_{i=1}^{n} \lambda_i + \lambda_{n+1} \left\langle \vec{v}_i + \vec{v}_{n+1}, \vec{w} \right\rangle = \sum_{i=1}^{n+1} \lambda_i \left\langle \vec{v}_i, \vec{w} \right\rangle $$

        Lo que implica que 

        $$\left\langle \sum_{i=1}^{n+1} \lambda_i \vec{v}_i, \vec{w} \right\rangle = \sum_{i=1}^{n+1} \lambda_i \left\langle \vec{v}_i, \vec{w} \right\rangle$$

    \end{enumerate}
    Por lo tanto el prodcuto interior es lineal en la primera entrada. 
    
\end{proof}

\end{document}