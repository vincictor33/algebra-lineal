\documentclass[12pt]{article}

\addtolength{\hoffset}{-2.25cm}
\addtolength{\textwidth}{4.5cm}
\addtolength{\voffset}{-2.5cm}
\addtolength{\textheight}{5cm}
\setlength{\parskip}{0pt}
\setlength{\parindent}{15pt}

\usepackage{amsthm}
\usepackage{amsmath}
\usepackage{amssymb}
\usepackage{enumitem}
\usepackage[colorlinks = true, linkcolor = blue, citecolor = blue, final]{hyperref}

\usepackage{graphicx}
\usepackage{multicol}
\usepackage{ marvosym }
\usepackage{wasysym}
\usepackage{tikz}
\usepackage{xcolor} 
\usepackage{CJKutf8}
\usepackage{tabularx}
\usepackage{float}
\usepackage{pgfplots}
\usepackage{fancyhdr}
\usepackage{amsfonts}
\usepackage{physics}
\usepackage{amsmath}
\usetikzlibrary{patterns}

\newcommand{\subscript}[2]{$#1 _ #2$}
\newcommand{\ds}{\displaystyle}
\newcommand\N{\ensuremath{\mathbb{N}}}
\newcommand\K{\ensuremath{\mathbb{K}}}
\newcommand\R{\ensuremath{\mathbb{R}}}
\newcommand\Z{\ensuremath{\mathbb{Z}}}
\renewcommand\O{\ensuremath{\emptyset}}
\newcommand\Q{\ensuremath{\mathbb{Q}}}
\newcommand\C{\ensuremath{\mathbb{C}}}
\DeclareMathOperator{\spn}{span}

\pgfplotsset{compat=1.18} 
\setlength{\parindent}{0in}

\pagestyle{empty}

\begin{document}

\thispagestyle{empty}

{\scshape Algebra Lineal} \hfill {\scshape \large Tarea II} \hfill {\scshape Victor Ortega}
 
\smallskip

\hrule

\bigskip

\bigskip

\theoremstyle{definition}
\newtheorem*{definition}{Definición}

\theoremstyle{definition}
\newtheorem*{dem}{Demostración}

\theoremstyle{definition}
\newtheorem*{notation}{Notación}

\theoremstyle{definition}
\newtheorem*{theorem}{Teorema}

\theoremstyle{definition}
\newtheorem*{lema}{Lema}

\theoremstyle{remark}
\newtheorem*{observation}{Observación}

\theoremstyle{remark}
\newtheorem*{example}{Ejemplo}


\begin{enumerate} 

\item Sean ($V,+,\cdot$) un $\K$-espacio vectorial y $W \subseteq V \implies$

    \begin{equation*}
        \sum_{i=1}^{n}  {\lambda}_{i}{\vec{v}}_{i} =  {\lambda}_{1}{\vec{v}}_{1}+...+ {\lambda}_{n}{\vec{v}}_{n} \in W
    \end{equation*}

    $\forall \: {\vec{v}}_{1},...,{\vec{v}}_{n} \in W$ y $\forall \: {\lambda}_{1},...,{\lambda}_{n} \in \K$

    \begin{proof}
    Antes de comenzar, se enunciará un lema, demostrado en clase

    \begin{lema}
        Sea ($V, +, \cdot$) un $\K$-ésimo espacio vectorial y $W \subseteq V \implies$  ($W,+ \restriction_W,\cdot  {\restriction}_{W}$) es un subespacio vectorial de $V \iff$ se cumplen

    \begin{enumerate}[label={\roman*})]
        \item ${\vec{0}}_{V} \in W$ \hfill \textcolor{blue}{Cero en Subespacio}
        \item $\forall \: \vec{v}, \vec{u} \in W \implies \vec{v} + \vec{u} \in W $ \hfill \textcolor{blue}{Cerrado Bajo Suma}
        \item $\forall \: \vec{v} \in W $ y $\forall \: \lambda \in \K \implies \lambda \cdot \vec{v} \in W$ \hfill \textcolor{blue}{Cerrado Bajo Producto}
    \end{enumerate}
    \end{lema}
    
    La prueba se hace por inducción sobre $n$

    \begin{enumerate}
        \item $n=1$ Sea ${\lambda}_{1} \in\K$ y ${\vec{v}}_{1} \in W$

        Por el \textbf{Lema}, en particular el inciso iii) $\implies {\lambda}_{1}{\vec{v}}_{1} \in W $

        \item $n=2$ Sean ${\lambda}_{1},{\lambda}_{2} \in\K$ y ${\vec{v}}_{1},{\vec{v}}_{2} \in W$

        Por el \textbf{Lema}, en particular el inciso iii) $\implies {\lambda}_{1}{\vec{v}}_{1} \in W$ y ${\lambda}_{2}{\vec{v}}_{2} \in W$

        Por el inciso ii) $\implies {\lambda}_{1}{\vec{v}}_{1} + {\lambda}_{2}{\vec{v}}_{2} \in W$

        \item Supongamos cierto el resultado para $n=k$

        \item Sean ${\vec{v}}_{1},...,{\vec{v}}_{k},{\vec{v}}_{k+1} \in W$ y ${\lambda}_{1},...,{\lambda}_{k},{\lambda}_{k+1} \in \K$

        Veamos que 

        \begin{equation*}
            {\lambda}_{1}{\vec{v}}_{1}+...+{\lambda}_{k}{\vec{v}}_{k}+{\lambda}_{k+1}{\vec{v}}_{k+1} = ({\lambda}_{1}{\vec{v}}_{1}+...+{\lambda}_{k}{\vec{v}}_{k})+{\lambda}_{k+1}{\vec{v}}_{k+1}
        \end{equation*}

        Por el supuesto que el resultado es cierto en $n=k$, la combinación lineal dentro del paréntesis $\in W$. El término restante también $\in W$, por el inciso ii) del \textbf{Lema}

        \begin{equation*}
            {\lambda}_{1}{\vec{v}}_{1}+...+{\lambda}_{k}{\vec{v}}_{k}+{\lambda}_{k+1}{\vec{v}}_{k+1} \in W
        \end{equation*}
    \end{enumerate}
\end{proof}

\item $\langle S \rangle$ es el subespacio más chico que contiene al conjunto $S$

\begin{proof}
     Para ver esta nota, supongamos que $W \leq V$ y $S \subseteq W$ P.D. $\langle S \rangle \subseteq W$

    Sea $\vec{z} \in \langle S \rangle \implies \: \exists \: {\vec{v}}_{1},...,{\vec{v}}_{n} \in S$ y $\lambda_1,...,\lambda_n \in \K \backepsilon$

    \begin{equation*}
        \vec{z} = \sum_{i=1}^{n} \lambda_i \vec{v_i}
    \end{equation*}

    Como $S \subseteq W \implies {\vec{v}}_{1},...,{\vec{v}}_{n} \in W$
\end{proof}


\end{enumerate}

\end{document}